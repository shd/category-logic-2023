\documentclass[10pt]{article}

\pagestyle{plain}
\textwidth=15.5cm
\textheight=24.0cm
\oddsidemargin=0.3cm
\evensidemargin=0.5cm

\usepackage[utf8]{inputenc}
\usepackage[T2A]{fontenc}
\usepackage[russian]{babel}
\usepackage{amssymb,amsmath}
\usepackage{amsthm}
\usepackage{mathrsfs}
\usepackage{graphicx}
\usepackage[a4paper,bmargin=2.5cm]{geometry}
\usepackage{framed}
\usepackage{hyperref}
\usepackage{tikz}

\newtheorem{Th}{Теорема}
\newtheorem{Con}{Следствие}[Th]
\newtheorem{Lm}{Лемма}[Th]
\newtheorem{Rm}{Замечание}
\theoremstyle{definition}
\newtheorem{Def}{Определение}
\newtheorem{Pm}{Задача}[subsection]
% theorem counter resets every \subsection
\renewcommand{\thePm}{\arabic{Pm}}% Remove subsection from theorem counter representation

\DeclareMathOperator{\Aut}{Aut}
\DeclareMathOperator{\dom}{dom}
\DeclareMathOperator{\cod}{cod}
\DeclareMathOperator{\Hom}{Hom}

\DeclareRobustCommand{\ArrowOf}{\text{\reflectbox{$\wr$}}}

\DeclareMathSymbol{\mhyphen}{\mathord}{AMSa}{"39}

\voffset=0pt
\headheight=0pt
\headsep=0pt

\begin{document}

\def\chap#1#2{\ \\ {\large\bf#1 \ | \ \tt\scshape#2} \par}

\ \vspace{-1cm}

{\bf
    \ \\
    \Large\centerline{\scshape Категорная логика}
}\normalsize

\subsection{Введение в теорию категорий}
\begin{Pm}
    Покажите, что если множества, моноиды, и предпорядки рассматривать как категории, то функторы между ними это то же, что и гомоморфизмы.
\end{Pm}

\begin{Pm}
    Опишите инициальные и терминальные объекты в категориях: $Cat$, $Top$ (категория всех топологических пространств), $Group$.
\end{Pm}

\begin{Pm}
    Рассмотрим множество $X$.
    Любое множество подмножеств $\Omega \subset \mathcal{P}(X)$ образует категорию (в которой объекты -- подмножества и стрелка между двумя объектами $X$,$Y$ есть в том случае, если $X \subseteq Y$).

    Что значит, что в этой категории есть инициальный или терминальный объекты?
\end{Pm}


\begin{Def} [Напоминание]
    Конкретной категорией называется такая категория, у которой каждый объект -- множество, и каждая стрелка -- теоретико-множественная функция.
\end{Def}

\begin{Def}
    Функтор $U: \mathcal{A} \to \mathcal{B}$ называется унивалентным (строгим), если для любой пары стрелок $f,g: X\rightrightarrows Y$, если $U(f) = U(g)$, то $f = g$.
    Другими словами, функтор унивалентен если он инъективен на $\Hom(X, Y)$ для любых объектов $X,Y$ в $\mathcal{A}$.
\end{Def}
\begin{Pm}
    Докажите, что у каждой конкретной категории $\mathcal{A}$ есть унивалентный функтор $\mathcal{A} \to Set$.
\end{Pm}

\begin{Pm}
    Покажите, что каждое действие моноида (группы) $M$ на множество можно представить как функтор из $M$ как категории в $Set$ (действие моноида $M$ на $X$ -- это гомоморфизм из $M$ в моноид перестановок множества $X$).
\end{Pm}

\begin{Pm}
    Убедитесь, что следующие отображения дают примеры функторов:
    \begin{enumerate}
        \item $\mathcal{P} : Set \to Set$, которое множеству $X$ сопоставляет множество его подмножеств $\mathcal{P}(X)$ и функции $f : X \to Y$ функцию $\mathcal{P}(f)(A) = f(A),\ A\subseteq X$;
              и $\mathcal{P} : Set \to Set^{op}$, как предыдущее, только функции $f: X \to Y$ оно сопоставляет $\mathcal{P}(f)(B) = f^{-1}(B),\ B\subseteq Y$.
              % \item $\mathcal{S} : Set \to Group$, отображает множество $X$ группу его перестановок $\mathcal{S}(X)$ (каким может быть отображение на стрелках?).
        \item $Ring \to Ring$, отображение кольцу $R$ кольцо многочленов $R[x]$; $Ring \to Ring$, отображение кольцу $R$ его кольцо квадратных матриц $M_{n,n}(R)$.
        \item $F\mhyphen Vect \to F\mhyphen Vect^{op}$, отображение векторному пространству $K$ его сопряженное $K^*$ ($F\mhyphen Vect$ -- категория векторных пространств над полем $F$).
    \end{enumerate}
\end{Pm}

\begin{Pm}
    Как можно следующие объекты представить в виде функтора?
    \begin{enumerate}
        \item Стрелка $f: X\to Y$ в произвольной категории $\mathcal{A}$
        \item Цепочка функций в $Sets$
              $X_0 \overset{f_1}{\to} X_1 \overset{f_2}{\to} X_2  \ldots \overset{f_n}{\to} X_{n}$.
        \item Бесконечная цепочка вложенных подмножеств $\mathbb{R}$
              $X_0 \subset X_1 \subset X_2 \ldots \subset X_n \subset \ldots$;
    \end{enumerate}
\end{Pm}

\begin{Pm}
    Рассмотрим категорию $\mathcal{C}$ и зафиксируем в ней объект $X$.
    Построим отображение на объектах $\Hom(X,-): C \to Set$, которое каждому объекту сопоставляет множество $\Hom(X, A)$.
    Теперь определим отображение на стрелках следующим образом: стрелке $f : A \to B$ сопоставим отображение между $\Hom(X,A)$ и $\Hom(X,B)$, отправляющее $g: X\to A$ в $fg : X \to B$.

    Покажите, что эти отображения дают функтор.
\end{Pm}
% \begin{Pm}
% Пусть $2$ это категория с двумя объектами и одной стрелкой между ними, $\cdot \rightarrow \cdot$ (петли опущены).
% Покажите, что функторы из $2$ в категорию $\mathcal{A}$ это в точности стрелки $\mathcal{A}$ и функции ''начало'' и ''конец'' можно рассматривать как функторы $\dom, \cod: \mathcal{A}^2 \rightrightarrows \mathcal{A}$.
% \end{Pm}

\subsection{Сопряженные функторы}
\href{https://drive.google.com/file/d/1kc4Tb_ImVhWP9aFOOLEC75KcX9R62uhI/view?usp=share_link}{\underline{Конспект.}}
\begin{Pm}
    Рассмотрим множества с предпорядком $\mathcal A$ и $\mathcal B$ и ковариантное соответствие Галуа $(F, G)$.

    Доказать:

    \begin{enumerate}
        \item $a \leq GF(a)$
        \item $GFGF(a) \leq GF(a)$
        \item $a \leq a' \Rightarrow GF(a) \leq GF(a')$
    \end{enumerate}
\end{Pm}
\begin{Pm}
    Рассмотрим множества $X, Y$ и бинарное отношение $R \subseteq X \times Y$.

    Пусть $\mathcal A = (\mathcal P(x), \subseteq)$, $\mathcal B = (\mathcal P(y), \text{superseteq})$.

    Функторы $F : \mathcal A \rightarrow \mathcal B$ и $G : \mathcal B \rightarrow \mathcal A$ определены так:
    \begin{itemize}
        \item $A \subseteq X$. $F(A) = \{y \in Y | \forall x \in A. \ (x, y) \in R\}$
        \item $B \subseteq Y$. $F(B) = \{x \in X | \forall y \in B. \ (x, y) \in R\}$
    \end{itemize}

    Доказать (или опровергнуть), что $(F, G)$ -- соответствие Галуа.
\end{Pm}
\begin{Pm}
    Понять, как соотносятся сопряжение для множеств с предпорядками и сопряжение для категорий. Сопряжение для категорий можно взять в смысле четверки $(F, U, \eta, \epsilon)$.
\end{Pm}
\begin{Pm}
    Доказать утверждение.

    Сопряжение $(F, U, \eta, \epsilon)$ в категориях $\mathcal A, \mathcal B$ взаимно однозначно соответствует решению $(F, \eta, *)$ для функтора $U : \mathcal B \rightarrow \mathcal A$.

    План доказательства и само утверждение можно найти в книжке на странице 14.
\end{Pm}

\begin{Pm}
    Пусть $(F, G)$ -- соответствие Галуа между посетами (частично упорядоченными) $\mathcal A$ и $\mathcal B$. показать, что $F$ сохраняет супремумы, а $G$ сохраняет инфимумы. Доказать, что если $\mathcal A$ имеет, а $F$ сохраняет супремумы, то правое сопряжение $G : \mathcal B \rightarrow \mathcal A$ может быть вычислено формулой $G(b) = sup \{a \in \mathcal A | F(a) \leq b \}$.
\end{Pm}

\begin{Pm}
    Понять, объяснить, и доказать утверждение 3.4 на странице 15.
\end{Pm}

\begin{Pm}
    Пусть $\mathcal A$ и $\mathcal B$ - пред упорядоченные множества формул пропозиционального исчисления, где порядок -- следование. Для фиксированный формулы $C$ показать, что $F : \mathcal A \rightarrow \mathcal B$ и $G : \mathcal B \rightarrow \mathcal A$, определенные как $F(A) = C \land A$ и $G(B) = C \Rightarrow B$, являются парой сопряженных функторов. Что есть "единство противоположностей" в таком случае?
\end{Pm}

\begin{Pm}
    Пусть $\mathcal A = \mathcal B = \mathbf{Sets}$, $C$ -- фиксированное множество, $F(A) = C \times A$, $U(B) = B^C$ для всех $A$ и $B$. Расширить $U$ и $F$ до функторов и показать, что $U$ право-сопряжен к $F$.
\end{Pm}

\subsection{Пределы}

\begin{Pm}
    Найдите хотя бы два уравнителя функций \verb|id :: Int -> Int| и \verb|abs :: Int -> Int| в категории Hask.

    Объясните, почему следующие функции не являются уравнителями:
    \begin{enumerate}
        \item \verb|id :: Int -> Int|
        \item \verb|\n -> -n :: Int -> Int|
        \item \verb|\n -> 0 :: Int -> Int|
    \end{enumerate}
\end{Pm}

\begin{Pm}
    Найдите хотя бы один уравнитель функций \verb|fst :: (Int, Int) -> Int| и \\ \verb|snd :: (Int, Int) -> Int| в категории Hask.
\end{Pm}

\begin{Pm}
    Найдите хотя бы один уравнитель функций \verb|fst'| и \verb|snd'|, где:
    \begin{verbatim}
data IntOrChar = I Int | C Char

fst' :: (Int, Char) -> IntOrChar
fst' = I . fst

snd' :: (Int, Char) -> IntOrChar
snd' = C . snd
    \end{verbatim}

    в категории Hask.
\end{Pm}

\begin{Pm}
    Найдите хотя бы два коуравнителя функций \verb|id :: Int -> Int| и \verb|abs :: Int -> Int| в категории Hask.
\end{Pm}

\begin{Pm}
    Найдите хотя бы один pullback функций \verb|length :: [Int] -> Int| и \verb|length :: [Char] -> Int| в категории Hask.
    \label{pullback1}
\end{Pm}

\begin{Pm}
    Найдите хотя бы один pullback функций \verb|id :: Int -> Int| и \verb|(const 0) :: () -> Int| в категории Hask.
\end{Pm}

\begin{Pm}
    Найдите хотя бы один pullback функций \verb|(const ()) :: Int -> ()| и \verb|(const ()) :: Char -> ()| в категории Hask.
\end{Pm}

\begin{Pm}
    Опишите построение pullback'а из задачи \ref{pullback1} через двоичные произведения и уравнители.
\end{Pm}

\begin{Pm}
    Докажите, что любой предел можно построить через произведения и уравнители.
\end{Pm}

\begin{Pm}
    Найдите предел функтора \verb|Identity|.
\end{Pm}

\begin{Pm}
    Найдите предел функтора \verb|Maybe|.
\end{Pm}

\subsection{Декартово замкнутые категории}
\begin{Pm}
    Доказать, что $\mathbf{Grp}$ не является декартово замкнутой.
\end{Pm}

\begin{Pm}
    Доказать биективность \textasciicircum \;(преобразующей $g$ в $\hat{g}$)
\end{Pm}

\begin{Pm}
    $A$ и $B$ конечные множества из категории $\mathbf{Sets}$. Какова мощность $A^B$?
\end{Pm}

\begin{Pm}
    Узнать связь между $\Leftarrow$ и $\rightarrow$ в полурешетке Гейтинга
\end{Pm}

\subsection{Декартово замкнутые категории в уравнениях и графах}
\begin{Pm}
    Показать, что в любой декартовой категории:
    \begin{enumerate}
        \item $A \times 1 \cong A$
        \item $A \times B \cong B \times A$
        \item $(A \times B) \times C \cong A \times (B \times C)$
    \end{enumerate}
\end{Pm}

\begin{Pm}
    Показать, что в любой декартово замкнутой категории:
    \begin{enumerate}
        \item $A^1 \cong A$
        \item $1^A \cong 1$
        \item $(A \times B)^C \cong A^C \times B^C$
        \item $A^{B \times C} \cong (A^C)^B$
    \end{enumerate}
\end{Pm}

\begin{Pm}
    Записать эквивалентное определение декартово замкнутой категории через
    \begin{gather*}
        U_B = (\cdot)^B,\; F_B = (\cdot) \times B\; :\; \mathscr{A} \rightarrow \mathscr{A} \\
        \varepsilon_B(A) = \varepsilon_{A,B};\quad \varepsilon_B\; :\; F_B U_B \rightarrow 1_\mathscr{A} \\
        \eta_B(C) = \eta_{C,B}\; :\; C \rightarrow (C \times B)^B\text{,}
    \end{gather*}
    где $U_B(f) = f^B \equiv f \Leftarrow 1_B = (f\varepsilon_{A,B})^*$ для всех $f : A \rightarrow A'$ (см. предыдущую лекцию).
\end{Pm}

\begin{Pm}
    Доказать, что
    $$
        \ulcorner f \urcorner^\ArrowOf = f,\; \ulcorner g^\ArrowOf \urcorner = g\text{,}
    $$
    где $\ulcorner f \urcorner \equiv (f\pi'_{1,A})^*,\; f\; :\; A \rightarrow B$ и $g^\ArrowOf \equiv \varepsilon_{B,A} \langle g \bigcirc_A, 1_A \rangle,\; g\; :\; 1 \rightarrow B^A$.
\end{Pm}

\begin{Pm}
    Показать, что дедуктивная система $\mathscr{L}(x)$ с прошлой лекции~--- это $\mathscr{D}(\mathscr{L}_x)$, где $\mathscr{L}_x$~--- граф, полученный из $\mathscr{L}$ добавлением нового ребра $x$ между старыми вершинами $T$ и $A$.
\end{Pm}

\subsection{Топосы}
\begin{Pm}
    Топосы функторов. Покажите, что следующие категории можно представить как топос (указать классификатор подобъектов, доказать замкнутость):
    \begin{enumerate}
        \item $Set^2$
        \item $Set^\rightarrow$
        \item $Set^M$ -- категория действий моноида (группы).
    \end{enumerate}
\end{Pm}

\begin{Pm}
    Докажите теорему:
    для любого топоса $\mathcal{C}$ и любого его объекта $a$ категория стрелок над ним $\mathcal{C}\downarrow a$ является топосом.
\end{Pm}

\begin{Pm}
    Докажите, что в топосе $\mathcal{T}$ $f:A \to B$
    \begin{enumerate}
        \item мономорфизм тогда и только тогда, когда $\mathcal{T} \vDash \forall_{x\in A}\forall_{x^\prime \in A} (f x = f x^\prime) \Rightarrow x = x^\prime$
        \item эпиморифизм тогда и только тогда, когда $\mathcal{T} \vDash \forall_{y\in B}\exists_{x\in A}f x = y$
    \end{enumerate}
\end{Pm}

\begin{Def}
    Топос называется булевым, если он удовлетворяет формуле $\forall_{t\in \Omega}(t \lor \neg t)$.
\end{Def}
\begin{Pm}
    Докажите, что топос является булевым тогда и только тогда, когда
    $1 \overset{T}{\to} \Omega \overset{\bot}{\leftarrow} 1$ это диаграмма копроизведения.
\end{Pm}

\subsection{CAM}
\begin{Pm}
Константы Id, Fst, Snd, App --- а также одноместная операция $\Lambda$ и двуместные $\circ$ и $\langle\cdot,\cdot\rangle$ ---
имеют естественный смысл, задаваемый уравнениями ниже. Также заметим, что их можно понимать как некоторые стрелки 
в категории между объектами, понятными из контекста. Соответственно, операции --- это стрелки, строящиеся по другим стрелкам.
Поясните, что следующие уравнения эквиваленты уравнениям, задающим декартово-замкнутые категории 
(за исключением указаний объектов у констант и операторов и отсутствия уравнений для терминального объекта):
\vspace{0.3cm}

\begin{tabular}{ll}
(Ass) & $(x \circ y) \circ z = x \circ (y \circ z)$\\
(IdL) & $Id \circ x = x$\\
(IdR) & $x \circ Id = x$\\
(Fst) & $Fst \circ \langle x,y \rangle = x$\\
(Snd) & $Snd \circ \langle x,y \rangle = y$\\
(SPair) & $\langle Fst \circ x, Snd \circ x \rangle = x$\\
(App) & $App \circ \langle \Lambda(x) \circ Fst, Snd \rangle = x$\\
(S$\Lambda$) & $\Lambda(App \circ \langle x \circ Fst, Snd \rangle) = x$
\end{tabular}
\end{Pm}

\begin{Pm}
Покажите, что уравнения из предыдущего задания эквивалентны уравнениям Ass + IdL + IdR + Fst + Snd + DPair + Beta + D$\Lambda$ + AI + FSI, где:
\vspace{0.3cm}

\begin{tabular}{ll}
(DPair) & $\langle x \circ y \rangle \circ z = \langle x \circ z, y \circ z \rangle$\\
(Beta) & $App \circ \langle \Lambda(x), y \rangle = x \circ \langle Id, y \rangle$\\
(D$\Lambda$) & $\Lambda(x) \circ y = \Lambda(x \circ \langle y \circ Fst, Snd\rangle)$\\
(AI) & $\Lambda(App) = Id$\\
(FSI) & $\langle Fst, Snd \rangle = Id$
\end{tabular}

\end{Pm}

\begin{Pm}
Поясните, что делают операции, определяемые следующим образом:
\vspace{0.3cm}

\begin{tabular}{l}
$A ^> := \Lambda(A \circ Snd)$\\
$A ^< := App \circ \langle A, Id \rangle$\\
$A \cdot B := (A \circ B^>)^<$\\
$(A,B) := \langle A^>, B^> \rangle ^<$
\end{tabular}
\end{Pm}

\subsection{Переписывание графов при помощи частичных морфизмов}

\begin{Pm}
Оказывается, утверждение 5 \href{https://www.sciencedirect.com/science/article/pii/0304397584900215}{статьи} не подходит для переписывания графов выражений. Найти ошибку в примере с семинара при переписывании \texttt{a * 1} в \texttt{a} в выражении \texttt{f(a * 1 + 3)}.
\end{Pm}

\begin{Pm}
В \href{https://repository.kulib.kyoto-u.ac.jp/dspace/bitstream/2433/82095/1/0754-28.pdf}{другой статье} приводится уточнение условий утверждения 5 вместе с контрпримером (пример 3.17). При этом, кажется, $g$ не является частичным морфизмом в определениях из оригинальной статьи. Возможно ли исправить контрпример?
\end{Pm}

\begin{Pm}
Попробовать применить утверждение 5 для переписывания графов. Обозначим множество вершин графа $V$, а множество рёбер~--- $E$. Предлагается представить такой граф в определениях статьи следующим образом:
$G = E + V$~--- множество-носитель;
\begin{align*}
\operatorname{succ}(s) &= \varepsilon \text{, если $s \in V$,} \\
\operatorname{succ}(e) &= st \text{, если $e \in E$ с началом $s$ и концом $t$.}
\end{align*}

Произвести переписывание графа (перевести графы в новое представление, предложить частичные морфизмы $f$ и $g$, множества $A'$ и $D$ и функции $lab$ и $succ$ в $D$):

\begin{center}
\begin{tabular}{c c c}
\begin{tikzpicture}[main/.style = {draw, circle}]
\node[main] (a) {$a$};
\node[main] (c) [below left of=a] {$c$};
\node[main] (b) [below right of=c] {$b$};

\draw[->] (a) -- (c);
\draw[->] (c) -- (b);
\end{tikzpicture}
&
\begin{tikzpicture}[main/.style = {draw, circle}]
\node[main] (a) {$a$};
\node[main] (c) [below left of=a] {$c$};
\node[main] (b) [below right of=c] {$b$};
\node[main] (d) [left of=c] {$d$};

\draw[->] (a) -- (c);
\draw[->] (c) -- (b);
\draw[->] (a) -- (d);
\draw[->] (d) -- (b);
\end{tikzpicture}
&
\begin{tikzpicture}[main/.style = {draw, circle}] 
\node[main] (a) {$a$};
\node[main] (c) [below left of=a] {$c$};
\node[main] (b) [below right of=c] {$b$};

\node[main] (f) [above of=a]{$f$};
\node[main] (e) [below of=b]{$e$};

\draw[->] (a) -- (c);
\draw[->] (c) -- (b);
\draw[->] (a) -- (b);
\draw[->] (f) -- (a);
\draw[->] (b) -- (e);
\end{tikzpicture}
\\
паттерн ($A$) & замена ($B$) & граф ($C$)
\end{tabular}
\end{center}
\end{Pm}

\begin{Pm}
Переписать следующий паттерн:

\begin{center}
\begin{tabular}{c c c}
\begin{tikzpicture}[main/.style = {draw, circle}] 
\node[main] (a) {$a$};
\node[main] (b) [below left of=a] {$b$};
\node[main] (c) [below right of=a] {$c$};

\draw[->] (a) -- (b);
\draw[->] (b) -- (c);
\draw[->] (c) -- (a);
\end{tikzpicture}
&
\begin{tikzpicture}[main/.style = {draw, circle}] 
\node[main] (a) {$a$};
\node[main] (b) [below left of=a] {$b$};
\node[main] (c) [below right of=a] {$c$};

\draw[->] (a) -- (b);
\draw[->] (b) -- (c);
\end{tikzpicture}
&
\begin{tikzpicture}[main/.style = {draw, circle}] 
\node[main] (a) {$a$};
\node[main] (b) [below left of=a] {$b$};
\node[main] (c) [below right of=a] {$c$};
\node[main] (d) [above of=a] {$d$};

\draw[->] (a) -- (b);
\draw[->] (b) -- (c);
\draw[->] (c) -- (a);

\draw[->] (d) -- (a);
\end{tikzpicture}
\\
паттерн & замена & граф
\end{tabular}
\end{center}
\end{Pm}

\begin{Pm}
Что произойдёт при попытке переписать граф следующим образом:

\begin{center}
\begin{tabular}{c c c}
\begin{tikzpicture}[main/.style = {draw, circle}] 
\node[main] (a) {$a$};
\node[main] (b) [below of=a] {$b$};

\draw[->] (a) -- (b);
\end{tikzpicture}
&
\begin{tikzpicture}[main/.style = {draw, circle}] 
\node[main] (a) {$a$};
\end{tikzpicture}
&
\begin{tikzpicture}[main/.style = {draw, circle}] 
\node[main] (a) {$a$};
\node[main] (b) [below of=a] {$b$};
\node[main] (c) [below of=b] {$c$};

\draw[->] (a) -- (b);
\draw[->] (b) -- (c);
\end{tikzpicture}
\\
паттерн & замена & граф
\end{tabular}
\end{center}
\end{Pm}

\end{document}
