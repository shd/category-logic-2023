\documentclass[10pt]{article}

\pagestyle{plain}
\textwidth=15.5cm
\textheight=24.0cm
\oddsidemargin=0.3cm
\evensidemargin=0.5cm

\usepackage[utf8]{inputenc}
\usepackage[T2A]{fontenc}
\usepackage[russian]{babel}
\usepackage{amssymb,amsmath}
\usepackage{amsthm}
\usepackage{graphicx}
\usepackage[a4paper,bmargin=2.5cm]{geometry}
\usepackage{framed}
\usepackage{hyperref}

\newtheorem{Th}{Теорема} 
\newtheorem{Con}{Следствие}[Th] 
\newtheorem{Lm}{Лемма}[Th] 
\newtheorem{Rm}{Замечание}
\theoremstyle{definition}
\newtheorem{Def}{Определение} 
\newtheorem{Pm}{Задача}[subsection]
% theorem counter resets every \subsection
\renewcommand{\thePm}{\arabic{Pm}}% Remove subsection from theorem counter representation

\DeclareMathOperator{\Aut}{Aut}
\DeclareMathOperator{\dom}{dom}
\DeclareMathOperator{\cod}{cod}
\DeclareMathOperator{\Hom}{Hom}

\DeclareMathSymbol{\mhyphen}{\mathord}{AMSa}{"39}

\voffset=0pt
\headheight=0pt
\headsep=0pt

\begin{document}

\def\chap#1#2{\ \\ {\large\bf#1 \ | \ \tt\scshape#2} \par}

\ \vspace{-1cm}

{\bf
\ \\
\Large\centerline{\scshape Категорная логика}
}\normalsize

\subsection{Введение в теорию категорий}
\begin{Pm}
Покажите, что если множества, моноиды, и предпорядки рассматривать как категории, то функторы между ними это то же, что и гомоморфизмы.  
\end{Pm}

\begin{Pm}
    Опишите инициальные и терминальные объекты в категориях: $Cat$, $Top$ (категория всех топологических пространств), $Group$.
\end{Pm}

\begin{Pm}
    Рассмотрим множество $X$. 
    Любое множество подмножеств $\Omega \subset \mathcal{P}(X)$ образует категорию (в которой объекты -- подмножества и стрелка между двумя объектами $X$,$Y$ есть в том случае, если $X \subseteq Y$). 

    Что значит, что в этой категории есть инициальный или терминальный объекты?
\end{Pm}


\begin{Def} [Напоминание]
  Конкретной категорией называется такая категория, у которой каждый объект -- множество, и каждая стрелка -- теоретико-множественная функция.
\end{Def}

\begin{Def}
    Функтор $U: \mathcal{A} \to \mathcal{B}$ называется унивалентным (строгим), если для любой пары стрелок $f,g: X\rightrightarrows Y$, если $U(f) = U(g)$, то $f = g$.
    Другими словами, функтор унивалентен если он инъективен на $\Hom(X, Y)$ для любых объектов $X,Y$ в $\mathcal{A}$.
\end{Def}
\begin{Pm}
  Докажите, что у каждой конкретной категории $\mathcal{A}$ есть унивалентный функтор $\mathcal{A} \to Set$. 
\end{Pm}

\begin{Pm}
    Покажите, что каждое действие моноида (группы) $M$ на множество можно представить как функтор из $M$ как категории в $Set$ (действие моноида $M$ на $X$ -- это гомоморфизм из $M$ в моноид перестановок множества $X$).
\end{Pm}

\begin{Pm}
    Убедитесь, что следующие отображения дают примеры функторов:
    \begin{enumerate}
        \item $\mathcal{P} : Set \to Set$, которое множеству $X$ сопоставляет множество его подмножеств $\mathcal{P}(X)$ и функции $f : X \to Y$ функцию $\mathcal{P}(f)(A) = f(A),\ A\subseteq X$;
        и $\mathcal{P} : Set \to Set^{op}$, как предыдущее, только функции $f: X \to Y$ оно сопоставляет $\mathcal{P}(f)(B) = f^{-1}(B),\ B\subseteq Y$. 
        % \item $\mathcal{S} : Set \to Group$, отображает множество $X$ группу его перестановок $\mathcal{S}(X)$ (каким может быть отображение на стрелках?).
        \item $Ring \to Ring$, отображение кольцу $R$ кольцо многочленов $R[x]$; $Ring \to Ring$, отображение кольцу $R$ его кольцо квадратных матриц $M_{n,n}(R)$.   
        \item $F\mhyphen Vect \to F\mhyphen Vect^{op}$, отображение векторному пространству $K$ его сопряженное $K^*$ ($F\mhyphen Vect$ -- категория векторных пространств над полем $F$).
    \end{enumerate}
\end{Pm}

\begin{Pm}
    Как можно следующие объекты представить в виде функтора? 
    \begin{enumerate}
        \item Стрелка $f: X\to Y$ в произвольной категории $\mathcal{A}$
        \item Цепочка функций в $Sets$
        $X_0 \overset{f_1}{\to} X_1 \overset{f_2}{\to} X_2  \ldots \overset{f_n}{\to} X_{n}$.
        \item Бесконечная цепочка вложенных подмножеств $\mathbb{R}$ 
        $X_0 \subset X_1 \subset X_2 \ldots \subset X_n \subset \ldots$;
    \end{enumerate}
\end{Pm}

\begin{Pm}
    Рассмотрим категорию $\mathcal{C}$ и зафиксируем в ней объект $X$. 
    Построим отображение на объектах $\Hom(X,-): C \to Set$, которое каждому объекту сопоставляет множество $\Hom(X, A)$. 
    Теперь определим отображение на стрелках следующим образом: стрелке $f : A \to B$ сопоставим отображение между $\Hom(X,A)$ и $\Hom(X,B)$, отправляющее $g: X\to A$ в $fg : X \to B$. 

    Покажите, что эти отображения дают функтор. 
\end{Pm}
% \begin{Pm}
% Пусть $2$ это категория с двумя объектами и одной стрелкой между ними, $\cdot \rightarrow \cdot$ (петли опущены).  
% Покажите, что функторы из $2$ в категорию $\mathcal{A}$ это в точности стрелки $\mathcal{A}$ и функции ''начало'' и ''конец'' можно рассматривать как функторы $\dom, \cod: \mathcal{A}^2 \rightrightarrows \mathcal{A}$. 
% \end{Pm}

\subsection{Сопряженные функторы}
\href{https://drive.google.com/file/d/1kc4Tb_ImVhWP9aFOOLEC75KcX9R62uhI/view?usp=share_link}{\underline{Конспект.}}
\begin{Pm}
    Рассмотрим множества с предпорядком $\mathcal A$ и $\mathcal B$ и ковариантное соответствие Галуа $(F, G)$.

    Доказать:

    \begin{enumerate}
        \item $a \leq GF(a)$
        \item $GFGF(a) \leq GF(a)$
        \item $a \leq a' \Rightarrow GF(a) \leq GF(a')$
    \end{enumerate}
\end{Pm}
\begin{Pm}
    Рассмотрим множества $X, Y$ и бинарное отношение $R \subseteq X \times Y$.

    Пусть $\mathcal A = (\mathcal P(x), \subseteq)$, $\mathcal B = (\mathcal P(y), \text{superseteq})$.

    Функторы $F : \mathcal A \rightarrow \mathcal B$ и $G : \mathcal B \rightarrow \mathcal A$ определены так:
    \begin{itemize}
        \item $A \subseteq X$. $F(A) = \{y \in Y | \forall x \in A. \ (x, y) \in R\}$
        \item $B \subseteq Y$. $F(B) = \{x \in X | \forall y \in B. \ (x, y) \in R\}$
    \end{itemize}

    Доказать (или опровергнуть), что $(F, G)$ -- соответствие Галуа.
\end{Pm}
\begin{Pm}
    Понять, как соотносятся сопряжение для множеств с предпорядками и сопряжение для категорий. Сопряжение для категорий можно взять в смысле четверки $(F, U, \eta, \epsilon)$.
\end{Pm}
\begin{Pm}
    Доказать утверждение.

    Сопряжение $(F, U, \eta, \epsilon)$ в категориях $\mathcal A, \mathcal B$ взаимно однозначно соответствует решению $(F, \eta, *)$ для функтора $U : \mathcal B \rightarrow \mathcal A$.

    План доказательства и само утверждение можно найти в книжке на странице 14.
\end{Pm}

\begin{Pm}
    Пусть $(F, G)$ -- соответствие Галуа между посетами (частично упорядоченными) $\mathcal A$ и $\mathcal B$. показать, что $F$ сохраняет супремумы, а $G$ сохраняет инфимумы. Доказать, что если $\mathcal A$ имеет, а $F$ сохраняет супремумы, то правое сопряжение $G : \mathcal B \rightarrow \mathcal A$ может быть вычислено формулой $G(b) = sup \{a \in \mathcal A | F(a) \leq b \}$.
\end{Pm}

\begin{Pm}
    Понять, объяснить, и доказать утверждение 3.4 на странице 15.
\end{Pm}

\begin{Pm}
    Пусть $\mathcal A$ и $\mathcal B$ - пред упорядоченные множества формул пропозиционального исчисления, где порядок -- следование. Для фиксированный формулы $C$ показать, что $F : \mathcal A \rightarrow \mathcal B$ и $G : \mathcal B \rightarrow \mathcal A$, определенные как $F(A) = C \land A$ и $G(B) = C \Rightarrow B$, являются парой сопряженных функторов. Что есть "единство противоположностей" в таком случае?
\end{Pm}

\begin{Pm}
    Пусть $\mathcal A = \mathcal B = \mathbf{Sets}$, $C$ -- фиксированное множество, $F(A) = C \times A$, $U(B) = B^C$ для всех $A$ и $B$. Расширить $U$ и $F$ до функторов и показать, что $U$ право-сопряжен к $F$.
\end{Pm}

\subsection{Пределы}

\begin{Pm}
    Найдите хотя бы два уравнителя функций \verb|id :: Int -> Int| и \verb|abs :: Int -> Int| в категории Hask.
    
    Объясните, почему следующие функции не являются уравнителями:
    \begin{enumerate}
        \item \verb|id :: Int -> Int|
        \item \verb|\n -> -n :: Int -> Int|
        \item \verb|\n -> 0 :: Int -> Int|
    \end{enumerate}
\end{Pm}

\begin{Pm}
    Найдите хотя бы один уравнитель функций \verb|fst :: (Int, Int) -> Int| и \\ \verb|snd :: (Int, Int) -> Int| в категории Hask.
\end{Pm}

\begin{Pm}
    Найдите хотя бы один уравнитель функций \verb|fst'| и \verb|snd'|, где:
    \begin{verbatim}
data IntOrChar = I Int | C Char

fst' :: (Int, Char) -> IntOrChar
fst' = I . fst

snd' :: (Int, Char) -> IntOrChar
snd' = C . snd
    \end{verbatim}
    
    в категории Hask.
\end{Pm}

\begin{Pm}
    Найдите хотя бы два коуравнителя функций \verb|id :: Int -> Int| и \verb|abs :: Int -> Int| в категории Hask.
\end{Pm}

\begin{Pm}
    Найдите хотя бы один pullback функций \verb|length :: [Int] -> Int| и \verb|length :: [Char] -> Int| в категории Hask.
    \label{pullback1}
\end{Pm}

\begin{Pm}
    Найдите хотя бы один pullback функций \verb|id :: Int -> Int| и \verb|(const 0) :: () -> Int| в категории Hask.
\end{Pm}

\begin{Pm}
    Найдите хотя бы один pullback функций \verb|(const ()) :: Int -> ()| и \verb|(const ()) :: Char -> ()| в категории Hask.
\end{Pm}

\begin{Pm}
    Опишите построение pullback'а из задачи \ref{pullback1} через двоичные произведения и уравнители.
\end{Pm}

\begin{Pm}
    Докажите, что любой предел можно построить через произведения и уравнители.
\end{Pm}

\begin{Pm}
    Найдите предел функтора \verb|Identity|.
\end{Pm}

\begin{Pm}
    Найдите предел функтора \verb|Maybe|.
\end{Pm}

\end{document}
